%! TeX root = ../main.tex

\chapter{Derivations}

\section{Fourier transform of the retarded correlator}%
\label{app:fourier-transform}

In this section we perform the Fourier transform of
the retarded correlator given in \zcref{eq:correlator-time},
and show how to arrive at \zcref{eq:correlator-frequency}.
As the transformation is linear,
we can transform each term separately:
\begin{align}
    C_\omega
     & =
    \lim_{\delta\to0^+}
    -\mi \int_0^\infty\! \md t\> \me^{\mi (\omega + \mi\delta) t}
    \langle A(t) B\rangle + \langle B A(t)\rangle \\
     & =
    \lim_{\delta\to0^+}
    -\mi \int_0^\infty\! \md t\> \me^{\mi (\omega + \mi\delta) t}
    \langle A(t) B\rangle
    +
    \lim_{\delta\to0^+}
    -\mi \int_0^\infty\! \md t\> \me^{\mi (\omega + \mi\delta) t}
    \langle B A(t)\rangle                         \\
     & =
    C^+_\omega + C^-_\omega.
\end{align}
We define the complex frequency $z=\omega+\mi\delta$ and take the limit $\delta\to0^+$ at the end.
Transformation of the first component:
\begin{align}
    C^+_z
     & =
    -\mi \int_0^\infty \md t \>
    \me^{\mi zt} \expval*{\me^{\mi Ht} A \me^{-\mi Ht} B}          \\
     & =
    -\mi \int_0^\infty \md t \>
    \me^{\mi zt} \expval*{\me^{\mi E_0t} A \me^{-\mi Ht} B}        \\
     & =
    -\mi \int_0^\infty \md t \>
    \expval{A \me^{\mi(z - H + E_0)t} B}                           \\
     & =
    -\mi \left.\expval*{A \frac{\me^{\mi(z - H + E_0)t}}{\mi(z - H + E_0)} B}
    \right|_0^\infty                                               \\
     & =
    -\mi \left(0 - \expval*{A \frac{1}{\mi(z - H + E_0)} B}\right) \\
     & =
    \expval*{A \frac{1}{z - (H - E_0)} B}
\end{align}
Taking the limit gives
\begin{equation}
    C^+_\omega
    =
    \lim_{\delta\to0^+} \expval*{A \frac{1}{z - (H - E_0)} B}
    =
    \expval*{A \frac{1}{\omega + \mi0^+ - (H - E_0)} B}.
\end{equation}
Transformation of the second component:
\begin{align}
    C^-_z
     & =
    -\mi \int_0^\infty \md t \>
    \me^{\mi zt} \expval*{B \me^{\mi Ht} A \me^{-\mi Ht}}   \\
     & =
    -\mi \int_0^\infty \md t \>
    \me^{\mi zt} \expval*{B \me^{\mi Ht} A \me^{-\mi E_0t}} \\
     & =
    -\mi \int_0^\infty \md t \>
    \expval*{B \me^{\mi(z + H - E_0)t} A}                   \\
     & =
    \expval*{B \frac{1}{z + (H - E_0)} A}
\end{align}
Taking the limit gives
\begin{equation}
    C^-_\omega
    =
    \lim_{\delta\to0^+} \expval*{A \frac{1}{z + (H - E_0)} B}
    =
    \expval*{A \frac{1}{\omega + \mi0^+ + (H - E_0)} B}.
\end{equation}

\section{DMFT simplification}%
\label{app:dmft-simplification}

To emphasize function arguments,
in this section we will forego the compact notation with parameters as subscripts
and write them in brackets instead,
e.g.\ $G_\omega \to G(\omega)$.

In the limit of infinite dimensions, the self-energy is $\vec{k}$-independent
and the local lattice Green's function can be evaluated from
its non-interacting counterpart:
\begin{equation}
    G_\mloc(\omega)
    =
    G_{\mloc,0}(\omega + \mu - \mu_0 - \Sigma(\omega)).
\end{equation}
We then map the lattice onto our impurity Green's function $G_0(\omega)$.
Without a Coulomb interaction on the lattice this requires
\begin{align}
    [G_0(\omega)]^{-1}
     & =
    [G_{\mloc,0}(\omega)]^{-1} \\
     & =
    \omega + \mi0^+ + \mu_0 - \epsilon - \Delta_0(\omega),
\end{align}
whereas in the interacting case it can be written as
\begin{align}
    [G_0(\omega)]^{-1}
     & =
    [G_{\mloc}(\omega)]^{-1} + \Sigma(\omega)                                          \\
     & =
    [G_{\mloc,0}(\omega + \mu - \mu_0 - \Sigma(\omega))]^{-1} + \Sigma(\omega)         \\
     & =
    \omega + \mi0^+ + \mu - \epsilon - \Delta_0(\omega + \mu - \mu_0 - \Sigma(\omega)) \\
     & \stackrel{!}{=}
    \omega + \mi0^+ + \mu - \epsilon - \Delta(\omega).
\end{align}
Comparing the last two lines, we can immediately deduce
\begin{equation}
    \Delta(\omega) = \Delta_0(\omega + \mu - \mu_0 - \Sigma(\omega)).
\end{equation}
