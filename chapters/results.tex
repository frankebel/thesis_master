%! TeX root = ../main.tex

\chapter{Results}

In this chapter we show results for our DMFT solver.
The calculations were done using Julia~\cite{Bezanson2017} version 1.11 and 1.12.
The codebase is contained in two packages:
\texttt{Fermions.jl}~\cite{Wallerberger2022} implements
Hamiltonians, wave functions and their interplay.
The second package \texttt{DMFT.jl}\footnote{
    \url{https://github.com/frankebel/DMFT.jl}
}
\fe{
    Should it be moved to the organization \url{https://github.com/tuwien-cms}?
    I'm not part of the organization and don't have permissions.
    Also try to find a better package name?
}
contains the DMFT self-consistency loop with representation of correlators and self-energies.

\section{Ground state convergence}

Before we can calculate any correlators or expectation values,
we first need to find the ground state of our system.
To get a discrete hybridization function,
we create $N$ equidistant points $\epsilon_i$ in the interval $[-D, D]$ and
discretize our semicircular DOS\footnote{
    We need to multiply it by $D^2/4$ as dictated by \zcref{eq:hybridization-function-bethe}.
}
(\zcref{eq:dos-semicircle}) onto these points.
The weight of each point is given by
\begin{equation}
    V_i^2 = \int_{I_i} \md\omega\> A(\omega),
\end{equation}
with the interval endpoints are chosen to be halfway between neighbors
\begin{equation}
    I_i = \left[\frac{\epsilon_{i-1} + \epsilon_i}{2}, \frac{\epsilon_i + \epsilon_{i+1}}{2}\right].
\end{equation}
For the first and last point the lower and upper bound are set to
$-\infty$ and $+\infty$ respectively.
This asserts that the sum of all discrete weights equals the total continuous spectral weight.
This method is slightly different from~\cite{Lu2019},
where the intervals were set in such a way that each weight $V_i^2$ was equal.
Given the parameters $\{V_i, \epsilon_i\}$,
we can create $H_\mathrm{bath}$ and $H_\mathrm{imp-bath}$ of \zcref{eq:H-Anderson}.
For the impurity $H_\mathrm{imp}$ we then add a Coulomb interaction of $U=2D$ which fixes our
on-site energy to $\epsilon-\mu=-D$ for half-filling.
We can then iteratively find our ground state using the algorithm described in \zcref{sec:ground-state}.

\section{Performance analysis}

To calculate our correlators,
we have to create the block tridiagonal matrix $T$
using the aforementioned block Lanczos algorithm.

\begin{figure}[ht]
    \centering
    %! TeX root = ../../main.tex

\begin{tikzpicture}
    [
        node distance=0pt,
    ]
    \begin{groupplot}
        [
            group style=
                {
                    group name=plots,
                    group size=2 by 2,
                    horizontal sep=2mm,
                    vertical sep=2mm,
                    x descriptions at=edge bottom,
                    ylabels at=edge left,
                },
            width=0.45\textwidth,
            xmode=log,
            ymode=log,
            cycle multiindex* list={
                    Set1\nextlist%
                    mark list*\nextlist
                },
        ]

        \nextgroupplot
        [
            xmax=2e2,
            ymin=1e-3,
            ymax=3e3,
            ylabel={runtime (\unit{\second})},
            legend style={
                    at={($(0,0)+(1cm,1cm)$)},
                    legend columns=2,
                    anchor=center,
                    align=center,
                    /tikz/every even column/.append style={column sep=5mm},
                },
            legend entries=
                {
                    {$L=1$, $p=1$},
                    {$L=2$, $p=1$},
                    {$L=1$, $p=2$},
                    {$L=2$, $p=2$},
                },
            legend to name=named,
        ]
        \addplot+ table [x=krylov, y=time_L1_p1] {data/benchmark_krylov.csv};
        \addplot+ table [x=krylov, y=time_L2_p1] {data/benchmark_krylov.csv};
        \addplot+ table [x=krylov, y=time_L1_p2] {data/benchmark_krylov.csv};
        \addplot+ table [x=krylov, y=time_L2_p2] {data/benchmark_krylov.csv};
        \node
        [
            text width=1em,
            anchor=north west,
            inner ysep=-3pt,
        ]
        (c1) at (rel axis cs:0,1) {\subcaption{}\label{subfig:perf-time-krylov}};

        \nextgroupplot
        [
            xmax=1e3,
            ymin=1e-3,
            ymax=3e3,
            yticklabel={\empty},
        ]
        \addplot+ table [x=bath, y=time_L1_p1] {data/benchmark_bath.csv};
        \addplot+ table [x=bath, y=time_L2_p1] {data/benchmark_bath.csv};
        \addplot+ table [x=bath, y=time_L1_p2] {data/benchmark_bath.csv};
        \addplot+ table [x=bath, y=time_L2_p2] {data/benchmark_bath.csv};
        \node
        [
            text width=1em,
            anchor=north west,
            inner ysep=-3pt,
        ]
        at (rel axis cs:0,1) {\subcaption{}\label{subfig:perf-time-sites}};
        \coordinate (c2) at (rel axis cs:1,1); % top right

        \nextgroupplot
        [
            xmax=2e2,
            ymin=1e6,
            ymax=8e10,
            xlabel={Krylov steps $M$},
            ylabel={total allocations (\unit{\byte})},
        ]
        \addplot+ table [x=krylov, y=allocations_L1_p1] {data/benchmark_krylov.csv};
        \addplot+ table [x=krylov, y=allocations_L2_p1] {data/benchmark_krylov.csv};
        \addplot+ table [x=krylov, y=allocations_L1_p2] {data/benchmark_krylov.csv};
        \addplot+ table [x=krylov, y=allocations_L2_p2] {data/benchmark_krylov.csv};
        \node
        [
            text width=1em,
            anchor=north west,
            inner ysep=-3pt,
        ]
        at (rel axis cs:0,1) {\subcaption{}\label{subfig:perf-allocations-krylov}};

        \nextgroupplot[
            xmax=1e3,
            ymin=1e6,
            ymax=8e10,
            xlabel={bath sites $N$},
            yticklabel={\empty},
        ]
        \addplot+ table [x=bath, y=allocations_L1_p1] {data/benchmark_bath.csv};
        \addplot+ table [x=bath, y=allocations_L2_p1] {data/benchmark_bath.csv};
        \addplot+ table [x=bath, y=allocations_L1_p2] {data/benchmark_bath.csv};
        \addplot+ table [x=bath, y=allocations_L2_p2] {data/benchmark_bath.csv};
        \node
        [
            text width=1em,
            anchor=north west,
            inner ysep=-3pt,
        ]
        at (rel axis cs:0,1) {\subcaption{}\label{subfig:perf-allocations-sites}};

    \end{groupplot}

    \coordinate (c3) at ($(c1)!.5!(c2)$); % halfway between c1, c2
    \node [below] at (c3 |- current bounding box.south) {\pgfplotslegendfromname{named}};
\end{tikzpicture}

    \caption{
        foo bar
        \subref{subfig:perf-time-krylov},
        \subref{subfig:perf-time-sites},
        \subref{subfig:perf-allocations-krylov}
        \subref{subfig:perf-allocations-sites}
    }
\end{figure}

\section{Convergence}

\section{Mott transition}

\section{Comparison against \quanty}

\section{\texorpdfstring{CeRu$_4$Sn$_6$}{CeRu4Sn6}}
