%! TeX root = ../../main.tex

\chapter{Definitions}

\section{Anderson impurity model}

% TODO: reference DMFT
In DMFT (more in section TODO) a Hubbard model is mapped onto an Anderson impurity model (AIM)
which then needs to be solved.
In this work we will only look at the single impurity Anderson model (SIAM).
Its Hamiltonian can be split into an interacting and non-interacting part
$H = H_0 + H_\mathrm{int}$.
Denoting the impurity annihilators by $d_\sigma$ ($n_\sigma = d^\dag_\sigma d^{\pdag}_\sigma$)
and bath annihilators by $c_\sigma$
it can be written as
\begin{align}
    H_0
     & =
    \sum_{\pk\sigma} (\epsilon_{d,\sigma} - \mu) d^\dag_{\pk\sigma} d^{\pdag}_{\pk\sigma}
    +
    \sum_{k\sigma} \epsilon_{k\sigma}^{\pdag} c^\dag_{k\sigma} c^{\pdag}_{k\sigma}
    +
    \sum_{k\sigma} \left(V^{\ps}_{k\sigma} d^\dag_{\vphantom{k}\sigma} c^{\pdag}_{k\sigma}
    + V_{k\sigma}^* c^\dag_{k\sigma} d^{\pdag}_{\vphantom{k}\sigma} \right),
    \\
    H_\mathrm{int}
     & =
    U n_\uparrow n_\downarrow.
\end{align}
Here,
$\epsilon_{d,\sigma}$ denotes the impurity energy,
$\mu$ the chemical potential,
$\epsilon_{k\sigma}$ the energy levels of the bath,
and $V_{k\sigma}$ the hybridization between the bath and impurity.
A graphical representation is given in \Cref{subfig:geometry-star}.

\begin{figure}[ht]
    \centering
    \savebox{\imagebox}{%! TeX root = ../../main.tex

\begin{tikzpicture}
    [
        node distance=2mm,
    ]
    % draw sites
    \foreach \i [remember=\i as \lasti, evaluate=\i as \filling using -0.1*\i+0.9] in {1,...,7}
        {
            \ifnum\i=1
                \node [bath={\filling}, label=right:$l_\i$] (l\i) {};
            \else
                \node [bath={\filling}, label=right:$l_\i$] (l\i) [above=of l\lasti] {};
            \fi
        }
    \node [impurity={0.5}, label=left:$i$] (impurity) [left=5mm of l4] {};

    % connect sites
    \foreach \i in {1,...,7}
        {
            \draw (impurity.east) to [out=0,in=180,looseness=0.3] (l\i.west);
        }
\end{tikzpicture}
} % save for height alignment
    \begin{subfigure}{0.45\textwidth}
        \centering
        \usebox{\imagebox}
        \caption{}
        \label{subfig:geometry-star}
    \end{subfigure}
    \begin{subfigure}{0.45\textwidth}
        \centering
        \raisebox{\dimexpr0.5\ht\imagebox-0.5\height}
        {
            %! TeX root = ../../main.tex

\begin{tikzpicture}[
        node distance=2mm,
    ]
    % draw sites
    \node [impurity={0.5}, label=left:$i$] (impurity) {};
    \node [bath={0.5}, label=right:$b$] (mirror) [right=of impurity] {};
    \foreach \i [remember=\i as \lasti] in {1,...,3}
        {
            \ifnum\i=1
                \node [bath={1.0}, label=below:$v_1$] (v\i) [below=of mirror] {};
                \node [bath={0.0}, label=above:$c_1$] (c\i) [above=of mirror] {};
            \else
                \node [bath={1.0}, label=below:$v_\i$] (v\i) [right=of v\lasti] {};
                \node [bath={0.0}, label=above:$c_\i$] (c\i) [right=of c\lasti] {};
            \fi
        }

    % connect sites
    \draw (impurity.east) to (mirror.west);
    \draw (impurity.south) to [out=270,in=180] (v1.west);
    \draw (impurity.north) to [out=90,in=180] (c1.west);
    \draw (mirror.south) to (v1.north);
    \draw (mirror.north) to (c1.south);
    \foreach \i [remember=\i as \lasti (initially 1)] in {2,...,3}
        {
            \draw (v\lasti.east) -- (v\i.west);
            \draw (c\lasti.east) -- (c\i.west);
        }
\end{tikzpicture}

        }
        \caption{}
        \label{subfig:geometry-natural-orbitals}
    \end{subfigure}
    \caption{
        Anderson impurity model in different geometries.
        Impurity (square) $i$ interacting with bath sites (circle) $l_i$, $b$, $v_i$, $c_i$.
        Site occupation is indicated by fill level, hopping by lines connecting sites.
        \Cref{subfig:geometry-star} is the conventional ``star'' geometry;
        \Cref{subfig:geometry-natural-orbitals} the natural impurity orbitals geometry
        with a mirror site ($b$), filled ($v_i$), and empty chains ($c_i$).
        % (adapted from~\cite{Lu2014}, Fig.~1)
    }
\end{figure}

\section{Correlators}

Given the ground state $\ket{\psi_0}$,
the retarded correlator for two fermionic operators $A$, $B$ is defined as
\begin{align}
    C(t) = -\mi \Theta(t) \braket{\{A(t), B\}},
\end{align}
with the anticommutator $\{\cdot,\cdot\}$
and expectation value $\braket{\cdot} = \braket{\psi_0 | \cdot | \psi_0}$.

Using complex frequency $z = \omega + \mi \delta$ do a Fourier transform
\begin{equation}
    C(z) = \int_{-\infty}^\infty \md t\>C(t) \me^{\mi zt} \eqqcolon \lAngle A, B \rAngle_z
\end{equation}
which can be split in
\begin{subequations}
    \begin{align}
        C^+(z)
         & =
        \Braket{A \frac{1}{z - H + E_0} B}
        =
        \sum_n \frac{\braket{\psi_0 | A | n}\braket{n | B | \psi_0}}{z - E_n + E_0}
        \\
        C^-(z)
         & =
        \Braket{B \frac{1}{z + H - E_0} A}
        =
        \sum_n \frac{\braket{\psi_0 | B | n}\braket{n | A | \psi_0}}{z + E_n - E_0}
    \end{align}
\end{subequations}
using a complete set of states $H\ket{n} = E_n\ket{n}$.

\section{Natural impurity orbitals}

natural impurity orbitals
Hybridization function
\begin{align}
    \Delta(z) = \sum_{k\sigma} \frac{|V_{k\sigma}|^2}{z - \epsilon_{k\sigma}}
\end{align}

\section{Lanczos algorithm}

\begin{itemize}
    \item Lanczos
    \item Block Lanczos
\end{itemize}

\section{Self-energy}

\begin{itemize}
    \item Dyson equation, $\Sigma^\mathrm{D}$
    \item $\Sigma^\mathrm{FG}$
    \item $\Sigma^\mathrm{IFG}$
\end{itemize}

\section{DMFT}

\begin{itemize}
    \item Bethe lattice
\end{itemize}

\section{Configuration interaction}
