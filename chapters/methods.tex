%! TeX root = ../main.tex

\chapter{Methods}%
\label{ch:methods}

\subsection{Natural impurity orbitals}%
\label{sec:natural-impurity-orbitals}

To solve the Anderson impurity model,
we can use Slater determinants to decompose the wave function
\begin{equation}
    \ket{\psi} = \sum_{i=1}^M \alpha_i \ket{\alpha_i}
    \label{eq:slater-decomposition}
\end{equation}
using any orthonormal set of basis functions.
With the occupation number,
we can express the number of electrons in each spin sector as
the sum of impurity and bath electrons
\begin{equation}
    N_\sigma = n_{i,\sigma} + \sum_{k=1}^N n_{b,k\sigma}.
\end{equation}
For a fixed amount of electrons $N_\mathrm{tot} = N_\uparrow + N_\downarrow$,
a combination of
\begin{equation}
    M = \binom{N+1}{n_\uparrow} \cdot \binom{N+1}{n_\downarrow}
\end{equation}
determinants are necessary to fully describe the wave function.
For a half-filled system ($N_\uparrow = N_\downarrow = (N+1)/2$)
with $N+1=140$ sites, this is approximately $\num{9e81}$
exceeding the number of protons in the observable universe $\bigO{\num{e80}}$. % Eddington number
As this is not treatable numerically,
we need to truncate a certain number of determinants with the smallest probabilities.
To save computational resources,
we want to find a new basis in which the probabilities
$|\alpha_i^2|$ in \zcref{eq:slater-decomposition}
decay faster than those of the star geometry
(further details in~\cite{Bi2019}).

Given a spin-orbital basis set $\{\ket{\phi_{i\sigma}}\}$
with their respective creator $c_{i\sigma}^\dag$~\cite{Bi2019},
the density matrix of the ground state is defined as
\begin{equation}
    n_{ij\sigma} = \sandwich{\psi_0}{c_{i\sigma}^\dag c_{j\sigma}}{\psi_0}.
\end{equation}
A commonly employed basis in quantum chemistry are the \emph{natural orbitals},
where the orbitals $\ket{\phi_{i\sigma}}$ are chosen such that
the aforementioned density matrix is diagonal $n_{ij\sigma} = \delta_{ij}n_{ij\sigma}$.
However,
a naive implementation of the natural orbitals will inevitably mix the original impurity
and bath sites transforming the local Coulomb interaction (\zcref{eq:Anderson-interaction})
among multiple new sites,
losing locality~\cite{Lu2014,Lu2019}.
We therefore use the \emph{natural impurity orbitals}
which only rotates bath orbitals while keeping the impurity untouched.
This new geometry is shown in \zcref{subfig:geometry-natural-impurity-orbitals}.
The unchanged impurity $i$ is connected to a mirror site (labeled $m$) and two chains.
Following the convention of~\cite{Lu2014,Lu2019},
we call sites in the almost fully filled chain valence bath sites (labeled $v$),
and sites in the almost empty chain as conduction bath sites (labeled $c$).
Hopping between sites is again shown by lines.

\section{Restricted active space}%
\label{sec:ras}

\fe{NOTE:\@ I think RAS fits better than CI.}

Using the natural impurity orbitals (\zcref{sec:natural-impurity-orbitals}) allows us to
write the ground state in a basis which has optimal scaling behavior with respect to
the number of bath sites~\cite{Lu2019}.
However, the computational complexity still grows combinatorial with each added bath site.
To alleviate the scaling problem we employ the same restricted active space (RAS) method
as \citeauthor{Lu2019}~\cite{Lu2019}, which is sketched in \zcref{fig:aim-separation}.

\begin{figure}[ht]
    \centering
    %! TeX root = ../../main.tex

\begin{tikzpicture}[
        inner sep=1mm,
        node distance=4mm,
    ]
    % sites
    \node [impurity={0.5}] (impurity) {};
    \node [bath={0.5}] (mirror) [right=of impurity] {};
    \foreach \i [remember=\i as \lasti] in {1,...,6}
        {
            \ifnum\i=1
                \node [bath={0.85}] (v\i) [below=of mirror] {};
                \node [bath={0.15}] (c\i) [above=of mirror] {};
            \else
                \ifnum\i=2
                    \node [bath={0.9}] (v\i) [right=of v\lasti] {};
                    \node [bath={0.1}] (c\i) [right=of c\lasti] {};
                \else
                    \ifnum\i=4
                        \node [bath={1.0}] (v\i) [right=8mm of v\lasti] {};
                        \node [bath={0.0}] (c\i) [right=8mm of c\lasti] {};
                    \else
                        \node [bath={1.0}] (v\i) [right=of v\lasti] {};
                        \node [bath={0.0}] (c\i) [right=of c\lasti] {};
                    \fi
                \fi
            \fi
        }

    % labels (separate node for box fit below)
    \node (impuritylabel)[left=0mm of impurity] {$i$};
    \node (mirrorlabel)[right=0mm of mirror] {$m$};
    \node (v1label)[below=0mm of v1] {$v_1$};
    \node (v3label)[below=0mm of v3] {$v_{L\vphantom{+1}}$}; % phantom for bounding box size
    \node (v4label)[below=0mm of v4] {$v_{L+1}$};
    \node (c1label)[above=0mm of c1] {$c_1$};
    \node (c3label)[above=0mm of c3] {$c_{L\vphantom{+1}}$}; % phantom for bounding box size
    \node (c4label)[above=0mm of c4] {$c_{L+1}$};


    % connect sites
    \draw (impurity.east) to (mirror.west);
    \draw (impurity.south) to [out=270,in=180] (v1.west);
    \draw (impurity.north) to [out=90,in=180] (c1.west);
    \draw (mirror.south) to (v1.north);
    \draw (mirror.north) to (c1.south);
    \foreach \i [remember=\i as \lasti (initially 1)] in {2,...,6}
        {
            \draw (v\lasti.east) -- (v\i.west);
            \draw (c\lasti.east) -- (c\i.west);
        }

    % bounding box
    \node [
        draw,
        rectangle,
        rounded corners,
        line width=0.2pt,
        fit=(impuritylabel) (v3label) (c3label),
        label=left:$|\phi_I\rangle$,
    ] {};
    \node [
        draw,
        rectangle,
        rounded corners,
        line width=0.2pt,
        fit=(v4label) (v6) (c4label) (c6),
        label=right:$|\phi_{II}\rangle$,
    ] {};
\end{tikzpicture}

    \caption{
        Separation of the wave function into two components
        $\ket{\phi_I}$ and $\ket{\phi_{II}}$ denoted by two boxes.
    }%
    \label{fig:aim-separation}
\end{figure}

The full Hilbert space is projected onto a subspace that only contains
empty conduction sites $\ket{\mathbb{0}_c}$
and full valence sites $\ket{\mathbb{1}_v}$ for $l>L$.
The ground state can be written as
\begin{equation}
    \ket{\psi_0}
    =
    \ket{\phi_I} \otimes \ket{\mathbb{0}_c} \otimes \ket{\mathbb{1}_v}
    =
    \ket{\phi_I} \otimes \ket{\phi_{II}},
\end{equation}
where $\ket{\phi_I}$ is the exact ground state calculated by ED for some $L$
and $\ket{\phi_{II}} = \ket{\mathbb{0}_c} \otimes \ket{\mathbb{1}_v}$ is a reference state.

In the next step we loosen the restriction of empty conduction and full valence bath sites
in $\ket{\phi_{II}}$.
The $p=1$ projection allows one state to differ from the reference state:
Either one electron of $\ket{\phi_I}$ enters the empty conduction sites
or one electron from the full valence sites enters $\ket{\phi_I}$.
In the $p=2$ projection two states are allowed to differ.
This allows for one of the following possibilities for $\ket{\phi_{II}}$:
Two electrons in the conduction bath sites $c_l$,
two holes in the valence bath sites $v_l$,
or one electron in the conduction bath sites $c_l$
and one hole in the valence bath sites $v_l$ ($l\ge L$).

