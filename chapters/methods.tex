%! TeX root = ../main.tex

\chapter{Methods}%
\label{ch:methods}

\subsection{Natural impurity orbitals}%
\label{sec:natural-impurity-orbitals}

To solve the Anderson impurity model,
we can use Slater determinants to decompose the wave function
\begin{equation}
    \ket{\psi} = \sum_{i=1}^M \alpha_i \ket{\alpha_i}
    \label{eq:slater-decomposition}
\end{equation}
using any orthonormal set of basis functions.
With the occupation number,
we can express the number of electrons in each spin sector as
the sum of impurity and bath electrons
\begin{equation}
    N_\sigma = n_{i,\sigma} + \sum_{k=1}^N n_{b,k\sigma}.
\end{equation}
For a fixed amount of electrons $N_\mathrm{tot} = N_\uparrow + N_\downarrow$,
a combination of
\begin{equation}
    M = \binom{N+1}{n_\uparrow} \cdot \binom{N+1}{n_\downarrow}
\end{equation}
determinants are necessary to fully describe the wave function.
For a half-filled system ($N_\uparrow = N_\downarrow = (N+1)/2$)
with $N+1=140$ sites, this is approximately $\num{9e81}$
exceeding the number of protons in the observable universe $\bigO{\num{e80}}$. % Eddington number
As this is not treatable numerically,
we need to truncate a certain number of determinants with the smallest probabilities.
To save computational resources,
we want to find a new basis in which the probabilities
$|\alpha_i^2|$ in \zcref{eq:slater-decomposition}
decay faster than those of the star geometry
(further details in~\cite{Bi2019}).

Given a spin-orbital basis set $\{\ket{\phi_{i\sigma}}\}$
with their respective creator $c_{i\sigma}^\dag$~\cite{Bi2019},
the density matrix of the ground state is defined as
\begin{equation}
    n_{ij\sigma} = \sandwich{\psi_0}{c_{i\sigma}^\dag c_{j\sigma}}{\psi_0}.
\end{equation}
A commonly employed basis in quantum chemistry are the \emph{natural orbitals},
where the orbitals $\ket{\phi_{i\sigma}}$ are chosen such that
the aforementioned density matrix is diagonal $n_{ij\sigma} = \delta_{ij}n_{ij\sigma}$.
However,
a naive implementation of the natural orbitals will inevitably mix the original impurity
and bath sites transforming the local Coulomb interaction (\zcref{eq:Anderson-interaction})
among multiple new sites,
losing locality~\cite{Lu2014,Lu2019}.
We therefore use the \emph{natural impurity orbitals}
which only rotates bath orbitals while keeping the impurity untouched.
This new geometry is shown in \zcref{subfig:geometry-natural-impurity-orbitals}.
The unchanged impurity $i$ is connected to a mirror site (labeled $m$) and two chains.
Following the convention of~\cite{Lu2014,Lu2019},
we call sites in the almost fully filled chain valence bath sites (labeled $v$),
and sites in the almost empty chain as conduction bath sites (labeled $c$).
Hopping between sites is again shown by lines.

