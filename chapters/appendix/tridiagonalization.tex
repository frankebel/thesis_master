%! TeX root = ../../main.tex

\chapter{Tridiagonalization}

Proof that the diagonal entries vanish when a symmetric impurity Green's function
is represented as a continued fraction.

The impurity solver returns the Green's function as a sum of poles
\begin{align}
    \mathcal{G}_{\!\omega} = \sum_{i=1}^N \frac{|b_i|^2}{\omega + \mi0^+ - a_i}.
    \label{eq:impurity-greens-function}
\end{align}
Symmetry enforces that for each location $a_i$ there exists a pole with opposite sign
$a_j = -a_i$ with same weight $|b_j|^2 = |b_i|^2$.
This in turn sets all moments for odd $m$ to zero \todo{explain moment}
% TODO: explain moment
\begin{align}
    M^{(m)}_{\mathcal{G}}
     & =
    \sum_i a_i^m |b_i|^2                                       \\
     & =
    \sum_{a_i>0} a_i^m |b_i|^2 + \sum_{a_j<0} a_j^m |b_j|^2    \\
     & =
    \sum_{a_i>0} a_i^m |b_i|^2 + \sum_{a_i>0} (-a_i)^m |b_i|^2 \\
     & =
    \sum_{a_i>0} a_i^m |b_i|^2 - \sum_{a_i>0} a_i^m |b_i|^2    \\
     & =
    0.
\end{align}
% WARN: Overfull \hbox
In \cite[appendix B]{Lu2014}, \citeauthor{Lu2014} showed how to transform
\zcref{eq:impurity-greens-function} into a continued fraction representation
using the lanczos algorithm on the diagonal matrix
\begin{equation}
    A
    =
    \diag(a_1, a_2, \ldots, a_N)
\end{equation}
and starting vector
\begin{equation}
    \vec{v}_0 = (b_1, b_2, \ldots, b_N)^\intercal.
\end{equation}

We can prove this by showing that the $l$-th component ($l \in \{1, \ldots, N\}$)
of the $i$-th Lanczos vector
can be written as a weighted sum
of even/odd powers of the original locations $a_k$
\begin{equation}
    v_{i,l}
    =
    b_l\!\sum\limits_{j=0}^{\lfloor i/2\rfloor}
    \begin{cases}
        c_{2j} a_l^{2j},     & i \text{ even} \\
        c_{2j+1} a_l^{2j+1}, & i \text{ odd}.
    \end{cases}
\end{equation}
For the base case this can be easily verified by setting $c_0=1$
\begin{equation}
    v_{0,l}
    =
    b_l\!\sum\limits_{j=0}^{0} c_{2j} a_l^{2j}
    =
    b_l c_{0} a_l^0
    =
    b_l.
\end{equation}
For the induction step we first apply the matrix $A$ to get a new state
\begin{equation}
    \tilde{\tilde{\vec{v}}}_{i+1,l}
    =
    A \vec{v}_{i,l}
    =
    b_l\!\sum\limits_{j=0}^{\lfloor i/2\rfloor}
    \begin{cases}
        c_{2j} a_l^{2j+1},   & i \text{ even} \\
        c_{2j+1} a_l^{2j+2}, & i \text{ odd}.
    \end{cases}
\end{equation}
The coefficient $a_i^t$ is given by the overlap to the previous state
\begin{align}
    a_i^t
     & =
    \vec{v}_i^\dagger A \vec{v}_i                                                \\
     & =
    \vec{v}_i^\dagger \tilde{\tilde{\vec{v}}}_{i+1}                              \\
     & =
    \sum\limits_{k,j}
    \begin{cases}
        c_{2k}^* c_{2j} \sum\limits_l a_l^{2k+2j+1}|b_l|^2,     & i \text{ even} \\
        c_{2k+1}^* c_{2j+1} \sum\limits_l a_l^{2k+2j+3}|b_l|^2, & i \text{ odd}
    \end{cases}     \\
     & =
    \sum\limits_{k,j}
    \begin{cases}
        c_{2k}^* c_{2j}    M^{(2k+2j+1)}_{\mathcal{G}},  & i \text{ even} \\
        c_{2k+1}^* c_{2j+1} M^{(2k+2j+3)}_{\mathcal{G}}, & i \text{ odd}
    \end{cases} \\
     & =
    0
\end{align}
Here we use the fact that all odd moments vanish.
Therefore, the new orthogonal state only couples to the second to last Lanczos vector
keeping even and odd powers separate.
\begin{equation}
    b_i^t \vec{v}_{i+1}
    =
    \tilde{\vec{v}}_{i+1}
    =
    \tilde{\tilde{\vec{v}}}_{i+1} - b_{i-1}^t \vec{v}_{i-1}.
\end{equation}
