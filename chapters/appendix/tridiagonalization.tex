%! TeX root = ../../main.tex

\chapter{Tridiagonalization}

Proof that the diagonal entries vanish when a symmetric impurity Green's function
is represented as a continued fraction.

The impurity solver returns the Green's function as a sum of poles
\begin{align}
    \mathcal{G}_{\!\omega} = \sum_{i=1}^N \frac{|b_i|^2}{\omega - \epsilon_i + \mi0^+}.
    \label{eq:impurity-greens-function}
\end{align}
Symmetry enforces that each location appears pairwise $\pm\epsilon_i$
with the same weight $|b_i|^2$.
This in turn sets all moments for odd $m$ to zero \todo{explain moment}
% TODO: explain moment
\begin{align}
    M^{(m)}_{\mathcal{G}}
     & =
    \sum_i \epsilon_i^m |b_i|^2                                                            \\
     & =
    \sum_{\epsilon_i>0} \epsilon_i^m |b_i|^2 + \sum_{\epsilon_i<0} \epsilon_i^m |b_i|^2    \\
     & =
    \sum_{\epsilon_i>0} \epsilon_i^m |b_i|^2 + \sum_{\epsilon_i>0} (-\epsilon_i)^m |b_i|^2 \\
     & =
    \sum_{\epsilon_i>0} \epsilon_i^m |b_i|^2 - \sum_{\epsilon_i>0} \epsilon_i^m |b_i|^2    \\
     & =
    0.
\end{align}
% WARN: Overfull \hbox
In \cite[appendix B]{Lu2014}, \citeauthor{Lu2014} showed how to transform
\zcref{eq:impurity-greens-function} into a continued fraction representation
using the lanczos algorithm on the diagonal matrix
\begin{equation}
    A
    =
    \diag(\epsilon_1, \epsilon_2, \ldots, \epsilon_N)
\end{equation}
and starting vector
\begin{equation}
    \vec{v}_1 = (b_1, b_2, \ldots, b_N)^\intercal.
\end{equation}

We can prove our statement by showing that the $i$-th component ($i \in \{1, \ldots, N\}$)
of the $j$-th Lanczos vector
can be written as a weighted sum
of separated even/odd powers of the original locations $\epsilon_i$
\begin{align}
    (\vec{v}_1)_i
     & = b_i                                            \\
    (\vec{v}_2)_i
     & = b_i c_{2,1} \epsilon_i                         \\
    (\vec{v}_3)_i
     & = b_i ( c_{3,2}\epsilon_i^2 + c_{3,0})           \\
    (\vec{v}_4)_i
     & = b_i ( c_{4,3}\epsilon_i^3 + c_{4,1}\epsilon_0) \\
    (\vec{v}_j)_i
     & =
    b_i\sum\limits_{k=0}^{\lfloor j/2\rfloor}
    \begin{cases}
        c_{j,2k+1} \epsilon_i^{2k+1}, & j \text{ even} \\
        c_{j,2k} \epsilon_i^{2k},     & j \text{ odd}.
    \end{cases}
    \label{eq:statement-tridiagonalization}
\end{align}
Here, the coefficient $c_{j,l}$ is associated with the $j$-th Lanczos vector
and $l$-th power in $\epsilon$.
For the base case ($j=1$) we can verify the statement by setting $c_{1,0}=1$
\begin{equation}
    (\vec{v}_1)_i
    =
    b_i\!\sum\limits_{k=0}^{0} c_{1,2k} \epsilon_i^{2k}
    =
    b_i c_{1,0} \epsilon_i^0
    =
    b_i.
\end{equation}
For the induction step we first calculate the new vector
\begin{align}
    (\vec{u}_{j+1})_i
     & =
    (A \vec{v}_j - \beta_{j-1}\vec{v}_{j-1})_i \\
     & =
    b_i
    \left(
    \sum\limits_k
    \begin{cases}
        c_{j,2k+1} \epsilon_i^{2k+2} \\
        c_{j,2k} \epsilon_i^{2k+1}
    \end{cases}
    - \beta_{j-1}
    \sum\limits_l
    \begin{cases}
        c_{j-1,2l} \epsilon_i^{2l} \\
        c_{j-1,2l+1} \epsilon_i^{2l+1}
    \end{cases}
    \right)
\end{align}
For the second term we can do an index shift $l \rightarrow k+1$
in order to collect the same powers of $\epsilon_i$
\begin{equation}
    (\vec{u}_{j+1})_i
    =
    b_i
    \sum\limits_k
    \begin{cases}
        (c_{j,2k+1} - \beta_{j-1} c_{j-1,2k+2}) \epsilon_i^{2k+2}, & j \text{ even} \\
        (c_{j,2k} - \beta_{j-1} c_{j-1,2l+3}) \epsilon_i^{2k+1},   & j \text{ odd}.
    \end{cases}.
\end{equation}
This means that this new vector $\vec{u}_j$ still has the form of separated even/odd powers
as in the statement (\zcref{eq:statement-tridiagonalization}).
The Lanczos coefficient $\alpha_j$ is given by the overlap
\begin{equation}
    \alpha_j
    =
    \vec{v}_j^\dagger A \vec{v}_i
    =
    \vec{v}_j^\dagger \vec{u}_{j+1}.
\end{equation}
If the case of even $j$, odd powers of $\epsilon$ (given by $\vec{v}_j$)
are multiplied by even powers of $\epsilon$ (given by $\vec{u}_j$)
(for odd $j$ vice versa).
Hence, the inner product only contains a sum of odd powers $\epsilon$ which all vanish
($\alpha_j = 0$)
and Lanczos vectors never couple to their predecessor
keeping even and odd powers of $\epsilon$ separated.

\section{Eigenvalues}

The tridiagonal matrix is given by
\begin{equation}
    T_N
    =
    \begin{pmatrix}
        0       & \beta_1 &         &             &             \\
        \beta_1 & 0       & \beta_2 &             &             \\
                & \beta_2 & 0       & \ddots      &             \\
                &         & \ddots  & \ddots      & \beta_{N-1} \\
                &         &         & \beta_{N-1} & 0
    \end{pmatrix}.
\end{equation}
As $\tr(T_N) = 0$, all eigenvalues must sum up to $0$ as well.
We can define a matrix
\begin{equation}
    L = \diag(1, -1, 1, -1, \ldots)
\end{equation}
with the properties $L^2 = \mathbb{1}, L^{-1} = L^\intercal = L$ and $L T_N L = -T_N$.
Applying it to the eigenvalue equation gives
\begin{align}
    L(T_N \vec{v}   & = \lambda \vec{v})    \\
    LT_N LL\vec{v}  & = \lambda L\vec{v}    \\
    -T_N (L\vec{v}) & = \lambda (L\vec{v})  \\
    T_N (L\vec{v})  & = -\lambda (L\vec{v})
\end{align}
meaning each eigenvalue appears pairwise $\pm\lambda$.
This is precisely the statement with which we started:
Each pole location appears pairwise $\pm\epsilon_i$ with same weight $|b_i|^2$.
